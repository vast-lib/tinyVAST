% Options for packages loaded elsewhere
\PassOptionsToPackage{unicode}{hyperref}
\PassOptionsToPackage{hyphens}{url}
%
\documentclass[
]{article}
\usepackage{amsmath,amssymb}
\usepackage{iftex}
\ifPDFTeX
  \usepackage[T1]{fontenc}
  \usepackage[utf8]{inputenc}
  \usepackage{textcomp} % provide euro and other symbols
\else % if luatex or xetex
  \usepackage{unicode-math} % this also loads fontspec
  \defaultfontfeatures{Scale=MatchLowercase}
  \defaultfontfeatures[\rmfamily]{Ligatures=TeX,Scale=1}
\fi
\usepackage{lmodern}
\ifPDFTeX\else
  % xetex/luatex font selection
\fi
% Use upquote if available, for straight quotes in verbatim environments
\IfFileExists{upquote.sty}{\usepackage{upquote}}{}
\IfFileExists{microtype.sty}{% use microtype if available
  \usepackage[]{microtype}
  \UseMicrotypeSet[protrusion]{basicmath} % disable protrusion for tt fonts
}{}
\makeatletter
\@ifundefined{KOMAClassName}{% if non-KOMA class
  \IfFileExists{parskip.sty}{%
    \usepackage{parskip}
  }{% else
    \setlength{\parindent}{0pt}
    \setlength{\parskip}{6pt plus 2pt minus 1pt}}
}{% if KOMA class
  \KOMAoptions{parskip=half}}
\makeatother
\usepackage{xcolor}
\usepackage[margin=1in]{geometry}
\usepackage{color}
\usepackage{fancyvrb}
\newcommand{\VerbBar}{|}
\newcommand{\VERB}{\Verb[commandchars=\\\{\}]}
\DefineVerbatimEnvironment{Highlighting}{Verbatim}{commandchars=\\\{\}}
% Add ',fontsize=\small' for more characters per line
\usepackage{framed}
\definecolor{shadecolor}{RGB}{248,248,248}
\newenvironment{Shaded}{\begin{snugshade}}{\end{snugshade}}
\newcommand{\AlertTok}[1]{\textcolor[rgb]{0.94,0.16,0.16}{#1}}
\newcommand{\AnnotationTok}[1]{\textcolor[rgb]{0.56,0.35,0.01}{\textbf{\textit{#1}}}}
\newcommand{\AttributeTok}[1]{\textcolor[rgb]{0.13,0.29,0.53}{#1}}
\newcommand{\BaseNTok}[1]{\textcolor[rgb]{0.00,0.00,0.81}{#1}}
\newcommand{\BuiltInTok}[1]{#1}
\newcommand{\CharTok}[1]{\textcolor[rgb]{0.31,0.60,0.02}{#1}}
\newcommand{\CommentTok}[1]{\textcolor[rgb]{0.56,0.35,0.01}{\textit{#1}}}
\newcommand{\CommentVarTok}[1]{\textcolor[rgb]{0.56,0.35,0.01}{\textbf{\textit{#1}}}}
\newcommand{\ConstantTok}[1]{\textcolor[rgb]{0.56,0.35,0.01}{#1}}
\newcommand{\ControlFlowTok}[1]{\textcolor[rgb]{0.13,0.29,0.53}{\textbf{#1}}}
\newcommand{\DataTypeTok}[1]{\textcolor[rgb]{0.13,0.29,0.53}{#1}}
\newcommand{\DecValTok}[1]{\textcolor[rgb]{0.00,0.00,0.81}{#1}}
\newcommand{\DocumentationTok}[1]{\textcolor[rgb]{0.56,0.35,0.01}{\textbf{\textit{#1}}}}
\newcommand{\ErrorTok}[1]{\textcolor[rgb]{0.64,0.00,0.00}{\textbf{#1}}}
\newcommand{\ExtensionTok}[1]{#1}
\newcommand{\FloatTok}[1]{\textcolor[rgb]{0.00,0.00,0.81}{#1}}
\newcommand{\FunctionTok}[1]{\textcolor[rgb]{0.13,0.29,0.53}{\textbf{#1}}}
\newcommand{\ImportTok}[1]{#1}
\newcommand{\InformationTok}[1]{\textcolor[rgb]{0.56,0.35,0.01}{\textbf{\textit{#1}}}}
\newcommand{\KeywordTok}[1]{\textcolor[rgb]{0.13,0.29,0.53}{\textbf{#1}}}
\newcommand{\NormalTok}[1]{#1}
\newcommand{\OperatorTok}[1]{\textcolor[rgb]{0.81,0.36,0.00}{\textbf{#1}}}
\newcommand{\OtherTok}[1]{\textcolor[rgb]{0.56,0.35,0.01}{#1}}
\newcommand{\PreprocessorTok}[1]{\textcolor[rgb]{0.56,0.35,0.01}{\textit{#1}}}
\newcommand{\RegionMarkerTok}[1]{#1}
\newcommand{\SpecialCharTok}[1]{\textcolor[rgb]{0.81,0.36,0.00}{\textbf{#1}}}
\newcommand{\SpecialStringTok}[1]{\textcolor[rgb]{0.31,0.60,0.02}{#1}}
\newcommand{\StringTok}[1]{\textcolor[rgb]{0.31,0.60,0.02}{#1}}
\newcommand{\VariableTok}[1]{\textcolor[rgb]{0.00,0.00,0.00}{#1}}
\newcommand{\VerbatimStringTok}[1]{\textcolor[rgb]{0.31,0.60,0.02}{#1}}
\newcommand{\WarningTok}[1]{\textcolor[rgb]{0.56,0.35,0.01}{\textbf{\textit{#1}}}}
\usepackage{graphicx}
\makeatletter
\def\maxwidth{\ifdim\Gin@nat@width>\linewidth\linewidth\else\Gin@nat@width\fi}
\def\maxheight{\ifdim\Gin@nat@height>\textheight\textheight\else\Gin@nat@height\fi}
\makeatother
% Scale images if necessary, so that they will not overflow the page
% margins by default, and it is still possible to overwrite the defaults
% using explicit options in \includegraphics[width, height, ...]{}
\setkeys{Gin}{width=\maxwidth,height=\maxheight,keepaspectratio}
% Set default figure placement to htbp
\makeatletter
\def\fps@figure{htbp}
\makeatother
\setlength{\emergencystretch}{3em} % prevent overfull lines
\providecommand{\tightlist}{%
  \setlength{\itemsep}{0pt}\setlength{\parskip}{0pt}}
\setcounter{secnumdepth}{-\maxdimen} % remove section numbering
\ifLuaTeX
  \usepackage{selnolig}  % disable illegal ligatures
\fi
\usepackage{bookmark}
\IfFileExists{xurl.sty}{\usepackage{xurl}}{} % add URL line breaks if available
\urlstyle{same}
\hypersetup{
  pdftitle={Spatial modeling},
  pdfauthor={James T. Thorson},
  hidelinks,
  pdfcreator={LaTeX via pandoc}}

\title{Spatial modeling}
\author{James T. Thorson}
\date{}

\begin{document}
\maketitle

\begin{Shaded}
\begin{Highlighting}[]
\FunctionTok{library}\NormalTok{(tinyVAST)}
\FunctionTok{library}\NormalTok{(mgcv)}
\FunctionTok{library}\NormalTok{(fmesher)}
\FunctionTok{set.seed}\NormalTok{(}\DecValTok{101}\NormalTok{)}
\FunctionTok{options}\NormalTok{(}\StringTok{"tinyVAST.verbose"} \OtherTok{=} \ConstantTok{FALSE}\NormalTok{)}
\end{Highlighting}
\end{Shaded}

\texttt{tinyVAST} is an R package for fitting vector autoregressive
spatio-temporal (VAST) models using a minimal and user-friendly
interface. We here show how it can fit spatial autoregressive model. We
first simulate a spatial random field and a confounder variable, and
simulate data from this simulated process.

\begin{Shaded}
\begin{Highlighting}[]
\CommentTok{\# Simulate a 2D AR1 spatial process with a cyclic confounder w}
\NormalTok{n\_x }\OtherTok{=}\NormalTok{ n\_y }\OtherTok{=} \DecValTok{25}
\NormalTok{n\_w }\OtherTok{=} \DecValTok{10}
\NormalTok{R\_xx }\OtherTok{=} \FunctionTok{exp}\NormalTok{(}\SpecialCharTok{{-}}\FloatTok{0.4} \SpecialCharTok{*} \FunctionTok{abs}\NormalTok{(}\FunctionTok{outer}\NormalTok{(}\DecValTok{1}\SpecialCharTok{:}\NormalTok{n\_x, }\DecValTok{1}\SpecialCharTok{:}\NormalTok{n\_x, }\AttributeTok{FUN=}\StringTok{"{-}"}\NormalTok{)) )}
\NormalTok{R\_yy }\OtherTok{=} \FunctionTok{exp}\NormalTok{(}\SpecialCharTok{{-}}\FloatTok{0.4} \SpecialCharTok{*} \FunctionTok{abs}\NormalTok{(}\FunctionTok{outer}\NormalTok{(}\DecValTok{1}\SpecialCharTok{:}\NormalTok{n\_y, }\DecValTok{1}\SpecialCharTok{:}\NormalTok{n\_y, }\AttributeTok{FUN=}\StringTok{"{-}"}\NormalTok{)) )}
\NormalTok{z }\OtherTok{=}\NormalTok{ mvtnorm}\SpecialCharTok{::}\FunctionTok{rmvnorm}\NormalTok{(}\DecValTok{1}\NormalTok{, }\AttributeTok{sigma=}\FunctionTok{kronecker}\NormalTok{(R\_xx,R\_yy) )}

\CommentTok{\# Simulate nuissance parameter z from oscillatory (day{-}night) process}
\NormalTok{w }\OtherTok{=} \FunctionTok{sample}\NormalTok{(}\DecValTok{1}\SpecialCharTok{:}\NormalTok{n\_w, }\AttributeTok{replace=}\ConstantTok{TRUE}\NormalTok{, }\AttributeTok{size=}\FunctionTok{length}\NormalTok{(z))}
\NormalTok{Data }\OtherTok{=} \FunctionTok{data.frame}\NormalTok{( }\FunctionTok{expand.grid}\NormalTok{(}\AttributeTok{x=}\DecValTok{1}\SpecialCharTok{:}\NormalTok{n\_x, }\AttributeTok{y=}\DecValTok{1}\SpecialCharTok{:}\NormalTok{n\_y), }\AttributeTok{w=}\NormalTok{w, }\AttributeTok{z=}\FunctionTok{as.vector}\NormalTok{(z) }\SpecialCharTok{+} \FunctionTok{cos}\NormalTok{(w}\SpecialCharTok{/}\NormalTok{n\_w}\SpecialCharTok{*}\DecValTok{2}\SpecialCharTok{*}\NormalTok{pi))}
\NormalTok{Data}\SpecialCharTok{$}\NormalTok{n }\OtherTok{=}\NormalTok{ Data}\SpecialCharTok{$}\NormalTok{z }\SpecialCharTok{+} \FunctionTok{rnorm}\NormalTok{(}\FunctionTok{nrow}\NormalTok{(Data), }\AttributeTok{sd=}\DecValTok{1}\NormalTok{)}

\CommentTok{\# Add columns for multivariate and temporal dimensions}
\NormalTok{Data}\SpecialCharTok{$}\NormalTok{var }\OtherTok{=} \StringTok{"n"}
\NormalTok{Data}\SpecialCharTok{$}\NormalTok{time }\OtherTok{=} \DecValTok{2020}
\end{Highlighting}
\end{Shaded}

We next construct a triangulated mesh that represents our continuous
spatial domain

\begin{Shaded}
\begin{Highlighting}[]
\CommentTok{\# make mesh}
\NormalTok{mesh }\OtherTok{=} \FunctionTok{fm\_mesh\_2d}\NormalTok{( Data[,}\FunctionTok{c}\NormalTok{(}\StringTok{\textquotesingle{}x\textquotesingle{}}\NormalTok{,}\StringTok{\textquotesingle{}y\textquotesingle{}}\NormalTok{)], }\AttributeTok{cutoff =} \DecValTok{2}\NormalTok{ )}

\CommentTok{\# Plot it}
\FunctionTok{plot}\NormalTok{(mesh)}
\end{Highlighting}
\end{Shaded}

\includegraphics{spatial_files/figure-latex/unnamed-chunk-3-1.pdf}

Finally, we can fit these data using \texttt{tinyVAST}

\begin{Shaded}
\begin{Highlighting}[]
\CommentTok{\# fit model}
\NormalTok{out }\OtherTok{=} \FunctionTok{tinyVAST}\NormalTok{( }\AttributeTok{data =}\NormalTok{ Data,}
           \AttributeTok{formula =}\NormalTok{ n }\SpecialCharTok{\textasciitilde{}} \FunctionTok{s}\NormalTok{(w),}
           \AttributeTok{spatial\_graph =}\NormalTok{ mesh,}
           \AttributeTok{control =} \FunctionTok{tinyVASTcontrol}\NormalTok{(}\AttributeTok{getsd=}\ConstantTok{FALSE}\NormalTok{),}
           \AttributeTok{sem =} \StringTok{""}\NormalTok{ )}
\end{Highlighting}
\end{Shaded}

We can then calculate the area-weighted total abundance:

\begin{Shaded}
\begin{Highlighting}[]
\CommentTok{\# Predicted sample{-}weighted total}
\FunctionTok{integrate\_output}\NormalTok{(out, }\AttributeTok{newdata =}\NormalTok{ out}\SpecialCharTok{$}\NormalTok{data)}
\CommentTok{\#\textgreater{}            Estimate          Std. Error Est. (bias.correct) Std. (bias.correct) }
\CommentTok{\#\textgreater{}          {-}104.15036            29.90545          {-}104.15036                  NA}
\CommentTok{\# integrate\_output(out, apply.epsilon=TRUE )}
\CommentTok{\# predict(out)}

\CommentTok{\# True (latent) sample{-}weighted total}
\FunctionTok{sum}\NormalTok{( Data}\SpecialCharTok{$}\NormalTok{z )}
\CommentTok{\#\textgreater{} [1] {-}92.89517}
\end{Highlighting}
\end{Shaded}

\section{Percent deviance explained}\label{percent-deviance-explained}

We can compute deviance residuals and percent-deviance explained:

\begin{Shaded}
\begin{Highlighting}[]
\NormalTok{R1 }\OtherTok{=} \FunctionTok{sum}\NormalTok{( }\FunctionTok{residuals}\NormalTok{(out, }\AttributeTok{type=}\StringTok{"deviance"}\NormalTok{)}\SpecialCharTok{\^{}}\DecValTok{2}\NormalTok{ )}

\CommentTok{\# tinyVAST null model with just a single intercept}
\NormalTok{null }\OtherTok{=} \FunctionTok{tinyVAST}\NormalTok{( }\AttributeTok{data =}\NormalTok{ Data,}
                 \AttributeTok{formula =}\NormalTok{ n }\SpecialCharTok{\textasciitilde{}} \DecValTok{1}\NormalTok{ )}
\NormalTok{R0 }\OtherTok{=} \FunctionTok{sum}\NormalTok{( }\FunctionTok{residuals}\NormalTok{(null, }\AttributeTok{type=}\StringTok{"deviance"}\NormalTok{)}\SpecialCharTok{\^{}}\DecValTok{2}\NormalTok{ )}

\CommentTok{\# Percent deviance explained}
\DecValTok{1} \SpecialCharTok{{-}}\NormalTok{ R1}\SpecialCharTok{/}\NormalTok{R0}
\CommentTok{\#\textgreater{} [1] 0.5051624}
\end{Highlighting}
\end{Shaded}

We can then compare this with the PDE reported by \texttt{mgcv}

\begin{Shaded}
\begin{Highlighting}[]
\NormalTok{start\_time }\OtherTok{=} \FunctionTok{Sys.time}\NormalTok{()}
\NormalTok{mygam }\OtherTok{=} \FunctionTok{gam}\NormalTok{( n }\SpecialCharTok{\textasciitilde{}} \FunctionTok{s}\NormalTok{(w) }\SpecialCharTok{+} \FunctionTok{s}\NormalTok{(x,y), }\AttributeTok{data=}\NormalTok{Data ) }\CommentTok{\#}
\FunctionTok{Sys.time}\NormalTok{() }\SpecialCharTok{{-}}\NormalTok{ start\_time}
\CommentTok{\#\textgreater{} Time difference of 0.04387498 secs}
\FunctionTok{summary}\NormalTok{(mygam)}\SpecialCharTok{$}\NormalTok{dev.expl}
\CommentTok{\#\textgreater{} [1] 0.3517756}
\end{Highlighting}
\end{Shaded}

where this comparison shows that using the SPDE method in tinyVAST
results in higher percent-deviance-explained. This reduced performance
for splines relative to the SPDE method presumably arises due to the
reduced rank of the spline basis expansion, and the better match for the
Matern function (in the SPDE method) relative to the true (simulated)
exponential semivariogram.

It is then easy to confirm that mgcv and tinyVAST give (essentially)
identical PDE when switching tinyVAST to use the same bivariate spline
for space.

\begin{Shaded}
\begin{Highlighting}[]
\NormalTok{out\_reduced }\OtherTok{=} \FunctionTok{tinyVAST}\NormalTok{( }\AttributeTok{data =}\NormalTok{ Data,}
                        \AttributeTok{formula =}\NormalTok{ n }\SpecialCharTok{\textasciitilde{}} \FunctionTok{s}\NormalTok{(w) }\SpecialCharTok{+} \FunctionTok{s}\NormalTok{(x,y) )}
\NormalTok{R1\_reduced }\OtherTok{=} \FunctionTok{sum}\NormalTok{( }\FunctionTok{residuals}\NormalTok{(out\_reduced, }\AttributeTok{type=}\StringTok{"deviance"}\NormalTok{)}\SpecialCharTok{\^{}}\DecValTok{2}\NormalTok{ )}
\DecValTok{1} \SpecialCharTok{{-}}\NormalTok{ R1\_reduced}\SpecialCharTok{/}\NormalTok{R0}
\CommentTok{\#\textgreater{} [1] 0.3497174}
\end{Highlighting}
\end{Shaded}

\section{Visualize spatial response}\label{visualize-spatial-response}

\texttt{tinyVAST} then has a standard \texttt{predict} function:

\begin{Shaded}
\begin{Highlighting}[]
\FunctionTok{predict}\NormalTok{(out, }\AttributeTok{newdata=}\FunctionTok{data.frame}\NormalTok{(}\AttributeTok{x=}\DecValTok{1}\NormalTok{, }\AttributeTok{y=}\DecValTok{1}\NormalTok{, }\AttributeTok{time=}\DecValTok{1}\NormalTok{, }\AttributeTok{w=}\DecValTok{1}\NormalTok{, }\AttributeTok{var=}\StringTok{"n"}\NormalTok{) )}
\CommentTok{\#\textgreater{} [1] 0.3649899}
\end{Highlighting}
\end{Shaded}

and this is used to compute the spatial response

\begin{Shaded}
\begin{Highlighting}[]
\CommentTok{\# Prediction grid}
\NormalTok{pred }\OtherTok{=} \FunctionTok{outer}\NormalTok{( }\FunctionTok{seq}\NormalTok{(}\DecValTok{1}\NormalTok{,n\_x,}\AttributeTok{len=}\DecValTok{51}\NormalTok{),}
              \FunctionTok{seq}\NormalTok{(}\DecValTok{1}\NormalTok{,n\_y,}\AttributeTok{len=}\DecValTok{51}\NormalTok{),}
              \AttributeTok{FUN=}\NormalTok{\textbackslash{}(x,y) }\FunctionTok{predict}\NormalTok{(out,}\AttributeTok{newdata=}\FunctionTok{data.frame}\NormalTok{(}\AttributeTok{x=}\NormalTok{x,}\AttributeTok{y=}\NormalTok{y,}\AttributeTok{w=}\DecValTok{1}\NormalTok{,}\AttributeTok{time=}\DecValTok{1}\NormalTok{,}\AttributeTok{var=}\StringTok{"n"}\NormalTok{)) )}
\FunctionTok{image}\NormalTok{( }\AttributeTok{x=}\FunctionTok{seq}\NormalTok{(}\DecValTok{1}\NormalTok{,n\_x,}\AttributeTok{len=}\DecValTok{51}\NormalTok{), }\AttributeTok{y=}\FunctionTok{seq}\NormalTok{(}\DecValTok{1}\NormalTok{,n\_y,}\AttributeTok{len=}\DecValTok{51}\NormalTok{), }\AttributeTok{z=}\NormalTok{pred, }\AttributeTok{main=}\StringTok{"Predicted response"}\NormalTok{ )}
\end{Highlighting}
\end{Shaded}

\includegraphics{spatial_files/figure-latex/unnamed-chunk-10-1.pdf}

\begin{Shaded}
\begin{Highlighting}[]

\CommentTok{\# True value}
\FunctionTok{image}\NormalTok{( }\AttributeTok{x=}\DecValTok{1}\SpecialCharTok{:}\NormalTok{n\_x, }\AttributeTok{y=}\DecValTok{1}\SpecialCharTok{:}\NormalTok{n\_y, }\AttributeTok{z=}\FunctionTok{matrix}\NormalTok{(Data}\SpecialCharTok{$}\NormalTok{z,}\AttributeTok{ncol=}\NormalTok{n\_y), }\AttributeTok{main=}\StringTok{"True response"}\NormalTok{ )}
\end{Highlighting}
\end{Shaded}

\includegraphics{spatial_files/figure-latex/unnamed-chunk-10-2.pdf}

We can also compute the marginal effect of the cyclic confounder

\begin{Shaded}
\begin{Highlighting}[]
\FunctionTok{library}\NormalTok{(pdp)  }\CommentTok{\# approx = TRUE gives effects for average of other covariates}
\FunctionTok{library}\NormalTok{(lattice)}
\CommentTok{\#library(visreg)}

\CommentTok{\# compute partial dependence plot}
\NormalTok{Partial }\OtherTok{=} \FunctionTok{partial}\NormalTok{( }\AttributeTok{object =}\NormalTok{ out,}
                   \AttributeTok{pred.var =} \StringTok{"w"}\NormalTok{,}
                   \AttributeTok{pred.fun =}\NormalTok{ \textbackslash{}(object,newdata) }\FunctionTok{predict}\NormalTok{(object,newdata),}
                   \AttributeTok{train =}\NormalTok{ Data,}
                   \AttributeTok{approx =} \ConstantTok{TRUE}\NormalTok{ )}

\CommentTok{\# Lattice plots as default option}
\FunctionTok{plotPartial}\NormalTok{( Partial )}
\end{Highlighting}
\end{Shaded}

\includegraphics{spatial_files/figure-latex/unnamed-chunk-11-1.pdf}

\end{document}
